\chapter{Problem Statement --- Objective vs Subjective}

\section{Introduction}

While objective Image Quality Assessment (IQA) metrics aim to quantify visual quality computationally, they often fall short of capturing subjective human perception. This discrepancy becomes particularly pronounced in specialized domains such as Face Image Quality Assessment (FIQA) and steganography, where biases and perceptual nuances play a significant role.

Subjective methods, such as Mean Opinion Score (MOS) evaluations, are considered the gold standard for assessing perceptual quality. However, these methods are time-consuming, costly, and influenced by human biases. In contrast, objective metrics offer scalability and reproducibility but frequently fail to align with human evaluations, especially for facial images or images altered by steganography.
%(Jo et al., 2024).

This chapter explores the gap between objective and subjective IQA, emphasizing the challenges associated with FIQA.\@ It highlights the inherent difficulties of evaluating steganography-encoded facial images and provides the foundation for developing metrics that better align with human perception.

\subsection{Limitations of Objective Metrics}

Objective metrics are often insufficient when evaluating images containing subtle distortions introduced by steganography. Steganographic methods embed information in ways that are perceptually invisible but can still affect overall quality. Current metrics are not sensitive enough to these modifications, leading to discrepancies between their scores and subjective opinions.

Furthermore, the evaluation of facial images introduces additional complexities. Humans tend to focus on key facial features, such as the eyes, nose, and mouth, when judging image quality. These features disproportionately influence subjective opinions, yet many objective metrics fail to prioritize their importance.

\subsection{Visualization of the Discrepancy}

To illustrate this discrepancy, consider a steganography-encoded facial image where distortions are localized to non-salient regions. An objective metric like PSNR might yield a high-quality score due to minimal pixel-level differences. However, human observers may rate the same image lower due to noticeable artifacts in key facial regions. Figure ref to fig depicts this inconsistency, highlighting the gap between computational scores and human perception.

%\begin{figure}[h]
%    \centering
%    \includegraphics[width=0.8\textwidth]{example_discrepancy.png}
%    \caption{Example of discrepancy between objective metric (PSNR) and subjective evaluation for a steganography-encoded facial image. Artifacts in key facial regions significantly lower perceived quality.}
%    \label{fig:discrepancy_example}
%\end{figure}
(insert image: Example of discrepancy between objective metric (PSNR) and subjective evaluation for a steganography-encoded facial image. Artifacts in key facial regions significantly lower perceived quality.)

This visualization underscores the necessity of developing metrics that integrate human perceptual biases and prioritize regions critical to subjective evaluations. Such metrics would bridge the gap, enabling more accurate assessments of image quality in applications like FIQA and steganography.

\section{Difficulties with FIQA}

Face Image Quality Assessment (FIQA) is inherently more challenging than general IQA due to the unique perceptual and psychological factors associated with facial images. These challenges stem from both the characteristics of human visual perception and the diverse applications of facial images in secure documentation, biometrics, and identity verification.

\subsection{Bias in Human Perception}

Humans exhibit a natural tendency to judge facial images based on aesthetic and emotional factors, often unrelated to technical quality. Features such as symmetry, skin texture, and expressions can disproportionately influence subjective evaluations, introducing variability into Mean Opinion Score (MOS) data (Jo et al., 2024). Additionally, the presence of anomalies or perceived unattractiveness in facial images can bias observers, leading to lower scores even for technically high-quality images.

\subsection{Region-Specific Sensitivities}

Facial images are not assessed uniformly; observers focus more on specific regions like the eyes, nose, and mouth. These regions are critical for recognition and carry significant perceptual weight. Distortions in these areas are more noticeable and impactful than those in non-salient regions, making it difficult for conventional metrics to capture their significance (Byungho et al., 2025). This sensitivity necessitates the development of metrics that prioritize these key regions in their evaluations.

\subsection{Dataset Complexity}

Facial image datasets are inherently diverse, encompassing variations in pose, lighting, occlusions, and expressions. For example, the inclusion of images with challenging conditions such as poor lighting or partial occlusions complicates both subjective and objective evaluations. Furthermore, datasets designed for steganography introduce additional layers of complexity, as embedding capacities and distortion thresholds vary across methods like StegaStamp or FStega (Ponomarenko et al., 2015).

\subsection{Impact on Secure Applications}

In applications such as biometric identification or identity verification, the quality of facial images directly impacts system performance. Poor-quality images can lead to increased error rates, reduced user trust, and compromised security. Metrics that fail to account for human perceptual biases and critical facial regions risk misclassifying or overlooking degradations, undermining the reliability of these systems.

Addressing these difficulties requires a paradigm shift in FIQA, moving beyond traditional pixel-based metrics to approaches that incorporate perceptual and semantic considerations. The next chapter, \textit{Related Work}, explores existing metrics and methodologies, highlighting their limitations and the need for novel solutions tailored to FIQA and steganography.
