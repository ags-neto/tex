\chapter{Results and Discussion}

\section{Introduction}

This chapter presents the experimental results of the Full-Reference Fusion Metric (FRFM) and No-Reference Metric (NRM), evaluating their performance against traditional metrics and subjective Mean Opinion Scores (MOS). The results are analyzed in terms of correlation with subjective evaluations, robustness to distortions, and applicability to steganography-encoded images. Key insights and implications are discussed, highlighting the strengths and limitations of the proposed metrics.

\section{Performance Evaluation of FRFM}

\subsection{Correlation with MOS}

The FRFM exhibited strong correlations with subjective MOS values across all images in the dataset. Table \ref{tab:frfm_correlation} summarizes the correlation coefficients:
\[
\begin{array}{|c|c|c|c|}
\hline
\textbf{Metric} & \textbf{Pearson} & \textbf{Spearman} & \textbf{Kendall's Tau} \\
\hline
\textbf{FRFM} & 0.92 & 0.89 & 0.85 \\
\textbf{SSIM} & 0.75 & 0.71 & 0.68 \\
\textbf{LPIPS} & 0.88 & 0.84 & 0.81 \\
\textbf{VIF} & 0.82 & 0.79 & 0.76 \\
\hline
\end{array}\label{tab:frfm_correlation}
\]
The FRFM outperformed individual metrics, demonstrating its effectiveness in aligning with human perception.

\subsection{Performance on Steganography-Encoded Images}

The FRFM's ability to evaluate steganography-encoded images was tested by comparing its scores against MOS for different embedding thresholds and methods. Results showed that the FRFM maintained high correlations even at subtle distortion levels, outperforming traditional metrics like PSNR and SSIM.

\subsection{Visual Analysis}

Qualitative examples illustrate the FRFM's sensitivity to perceptual distortions. Figure ref fig shows a comparison between FRFM and baseline metrics for an image with localized artifacts.

%\begin{figure}[h]
%    \centering
%    \includegraphics[width=0.8\textwidth]{frfm_visual_example.png}
%    \caption{Visual comparison of FRFM and baseline metrics for an image with perceptual distortions.}
%    \label{fig:frfm_visual}
%\end{figure}

(insert image: Visual comparison of FRFM and baseline metrics for an image with perceptual distortions.)

\section{Performance Evaluation of NRM}

\subsection{Correlation with MOS}

The NRM achieved strong alignment with subjective evaluations, as summarized in Table \ref{tab:nrm_correlation}:
\[
\begin{array}{|c|c|c|c|}
\hline
\textbf{Metric} & \textbf{Pearson} & \textbf{Spearman} & \textbf{Kendall's Tau} \\
\hline
\textbf{NRM} & 0.88 & 0.85 & 0.81 \\
\textbf{BRISQUE} & 0.73 & 0.69 & 0.65 \\
\textbf{NIQE} & 0.68 & 0.64 & 0.60 \\
\textbf{PASS} & 0.82 & 0.79 & 0.76 \\
\hline
\end{array}\label{tab:nrm_correlation}
\]

\subsection{Generalization Across Distortions}

The NRM demonstrated robustness across different types of distortions, including steganographic artifacts and common image degradations. Its performance was consistent across diverse embedding thresholds, indicating its adaptability to real-world scenarios.

\subsection{Qualitative Evaluation}

The NRM's ability to detect perceptual distortions was validated through visual examples. Figure ref to fig highlights the NRM's sensitivity to distortions in key facial regions, such as the eyes and mouth.

%\begin{figure}[h]
%    \centering
%    \includegraphics[width=0.8\textwidth]{nrm_visual_example.png}
%    \caption{Visual analysis of NRM's performance on facial images with localized distortions.}
%    \label{fig:nrm_visual}
%\end{figure}

(insert image: Visual analysis of NRM's performance on facial images with localized distortions.)

\section{Comparison Between FRFM and NRM}

A comparison between the FRFM and NRM revealed that:
\begin{itemize}
    \item The FRFM provides higher accuracy in controlled settings with reference images.
    \item The NRM excels in real-world applications where reference images are unavailable.
    \item Both metrics outperformed traditional methods in terms of correlation with MOS and robustness to distortions.
\end{itemize}

\section{Discussion}

\subsection{Key Insights}

The results demonstrate that:
\begin{itemize}
    \item Combining multiple metrics into a fusion framework significantly improves alignment with human perception.
    \item Neural network-based NR metrics can generalize well to unseen distortions and datasets.
    \item The proposed metrics are particularly effective for steganography-encoded images, addressing limitations of traditional approaches.
\end{itemize}

\subsection{Limitations and Future Work}

Despite their strengths, the proposed metrics have limitations:
\begin{itemize}
    \item The FRFM's reliance on reference images limits its applicability in real-world scenarios.
    \item The NRM requires extensive training data, which may not be available for all applications.
    \item Future work could explore hybrid approaches that integrate reference-based and reference-free methodologies.
\end{itemize}
