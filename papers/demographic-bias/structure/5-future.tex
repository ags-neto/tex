\section{Conclusion and Future Work}

Beyond the evaluation of IQA metrics and fusion methods, this study analyzed the extent of observer demographic biases in MOS ratings. The ANOVA results presented in Section~\ref{sec:methodology} indicate that observer characteristics, including gender, ethnicity, and country of origin, significantly influenced MOS.\@ These findings suggest that FIQA methodologies should account for potential biases introduced by subjective assessments from diverse observer groups.

A promising direction for future research is the development of a fully automated, no-reference ICAO-compliant image quality assessment model. Our current approach relies on a fusion of full-reference IQA metrics, which requires access to pristine reference images for quality comparison. However, in real-world applications, such as passport verification and ID issuance, reference images are often unavailable. To address this limitation, we propose training a deep neural network (DNN) on the fusion metric, enabling the transition from a reference-based to a no-reference image quality model.

Developing a deep learning-based, no-reference ICAO-compliant IQA model represents a crucial step toward improving image quality evaluation in biometric and security applications. This approach would ensure robust, unbiased, and practical quality assessments for automated identity verification systems.