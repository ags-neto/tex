\section{Related Work}

Significant research efforts have been devoted to improving the accuracy and fairness of facial image quality assessment (FIQA). The limitations of traditional image quality assessment (IQA) techniques in predicting perceptual quality have led to the development of specialized FIQA metrics tailored to face recognition applications~\cite{Terhoerst2020, Babnik2022}. These methods attempt to bridge the gap between objective quality measures and human perception by integrating statistical and perceptual quality indicators.

Recent studies have demonstrated that many FIQA methods exhibit demographic biases, where variations in ethnicity, age, and gender impact quality predictions. Cavazos et al.~\cite{Cavazos2021} analyzed multiple face recognition systems and found that lower recognition accuracy is often associated with underrepresented demographic groups, highlighting the limitations of IQA models trained on imbalanced datasets. Similarly, Kabbani et al.~\cite{Kabbani2024} observed that FIQA scores tend to favor certain demographic profiles, resulting in biased quality evaluations that could affect real-world biometric verification.

In addition to demographic biases, prior work has investigated the correlation between IQA metrics and mean opinion scores (MOS) obtained from human evaluators. Studies such as those by Huang et al.~\cite{Huang2020} and Terhoerst et al.~\cite{Terhoerst2020} have shown that traditional IQA metrics, including peak signal-to-noise ratio (PSNR) and structural similarity index measure (SSIM), fail to capture perceptual distortions effectively. Consequently, novel deep-learning-based approaches, such as learned perceptual image patch similarity (LPIPS) and deep image structure and texture similarity (DISTS), have been proposed to improve alignment with subjective evaluations~\cite{Babnik2022}.

Fusion-based IQA models have emerged as a promising solution to enhance MOS predictability. Recent work by Robinson et al.~\cite{Robinson2020} introduced a multi-metric fusion technique combining statistical, structural, and perceptual indicators to improve quality assessment consistency. Similarly, Kortylewski et al.~\cite{Kortylewski2019} proposed the use of synthetic data augmentation to mitigate biases in IQA model training. These studies suggest that integrating multiple quality metrics through machine learning or statistical fusion can lead to more robust assessments that align better with human perception.

Despite these advancements, there remains a need for a systematic evaluation of IQA fusion models to determine their efficacy in reducing bias and improving MOS correlation. This work builds upon prior research by conducting a comprehensive comparison of fusion-based IQA approaches, assessing their performance using Pearson's linear correlation coefficient (PLCC) and Spearman's rank-order correlation coefficient (SRCC). By analyzing the effectiveness of different fusion strategies, we aim to contribute towards the development of fairer and more accurate FIQA methodologies.
